\documentclass{article}

\title{Stat 601 - Lab 1}
\author{Anh Le}
\date{\today}

\begin{document}
\maketitle

n = 30:\\
\begin{tabular}{l c r}
Frequentist & Mean interval length & 0.278408 \\
& Mean coverage & 0.9466 \\
Uniform prior & Mean interval length & 0.2729083 \\
& Mean coverage & 0.9618 \\
\end{tabular}\\
When n is large, there is no difference between frequentist and uniform prior method. Both mean coverage is close to 0.95, as expected from a 95\% interval.\\

n = 5:\\
\begin{tabular}{l c r}
Frequentist & Mean interval length & 0.508149 \\
& Mean coverage & 0.659 \\
Uniform prior & Mean interval length & 0.5661504 \\
& Mean coverage & 0.9341 \\
\end{tabular}\\
When n is small, the uniform prior method gives much superior mean coverage (0.93) compared to frequentist (0.66). This confidence comes at the price of less precision: the mean interval length of Bayesian method is higher.\\

R code:\\
\begin{verbatim}
ci.freq.res1 = c()
ci.freq.res2 = c()
ci.unif.res1 = c()
ci.unif.res2 = c()

n = 30
for (i in 1:10000) {
  x = rbinom(1, n, 0.8)
  p.hat = x/n
  ci.freq.error = qnorm(1-0.05/2)*sqrt(p.hat*(1-p.hat)/n)
  ci.freq.lower = p.hat - ci.freq.error
  ci.freq.upper = p.hat + ci.freq.error
  ci.freq.length = 2*ci.freq.error
  ci.freq.true = (0.8 > ci.freq.lower) & (0.8 < ci.freq.upper)
  ci.freq.res1 = c(ci.freq.res1, ci.freq.length)
  ci.freq.res2 = c(ci.freq.res2, ci.freq.true)

  ci.unif.lower = qbeta(0.025, x+1, n-x+1)
  ci.unif.upper = qbeta(1-0.025, x+1, n-x+1)
  ci.unif.length = ci.unif.upper - ci.unif.lower
  ci.unif.true = (0.8 > ci.unif.lower) & (0.8 < ci.unif.upper)
  ci.unif.res1 = c(ci.unif.res1, ci.unif.length)
  ci.unif.res2 = c(ci.unif.res2, ci.unif.true)
}

mean(ci.freq.res1)
mean(ci.freq.res2)

mean(ci.unif.res1)
mean(ci.unif.res2)
\end{verbatim}
\end{document}
